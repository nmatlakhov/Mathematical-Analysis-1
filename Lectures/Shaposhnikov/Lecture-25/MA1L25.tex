\documentclass[12pt]{article}
\usepackage[left=1cm, right=1cm, top=2cm,bottom=1.5cm]{geometry} 

\usepackage[parfill]{parskip}
\usepackage[utf8]{inputenc}
\usepackage[T2A]{fontenc}
\usepackage[russian]{babel}
\usepackage{enumitem}
\usepackage[normalem]{ulem}
\usepackage{amsfonts, amsmath, amsthm, amssymb, mathtools}

\usepackage{accents}
\usepackage{fancyhdr}
\pagestyle{fancy}
\renewcommand{\headrulewidth}{1.5pt}
\renewcommand{\footrulewidth}{1pt}

\usepackage{graphicx}
\usepackage[figurename=Рис.]{caption}
\usepackage{subcaption}
\usepackage{float}

%%Наименование папки откуда забирать изображения
\graphicspath{ {./images/} }

%%Изменение формата для ввода доказательства
\renewcommand{\proofname}{$\square$  \nopunct}
\renewcommand\qedsymbol{$\blacksquare$}

\addto\captionsrussian{%
	\renewcommand{\proofname}{$\square$ \nopunct}%
}
%% Римские цифры
\newcommand{\RN}[1]{%
	\textup{\uppercase\expandafter{\romannumeral#1}}%
}

%% Для удобства записи
\newcommand{\MR}{\mathbb{R}}
\newcommand{\MQ}{\mathbb{Q}}
\newcommand{\MI}{\mathrm{I}}
\newcommand{\MJ}{\mathrm{J}}
\newcommand{\MH}{\mathrm{H}}
\newcommand{\MT}{\mathrm{T}}
\newcommand{\MU}{\mathcal{U}}
\newcommand{\MV}{\mathcal{V}}
\newcommand{\VN}{\varnothing}
\newcommand{\VE}{\varepsilon}

\theoremstyle{definition}
\newtheorem{defn}{Опр:}
\newtheorem{rem}{Rm:}
\newtheorem{prop}{Утв.}
\newtheorem{exrc}{Упр.}
\newtheorem{lemma}{Лемма}
\newtheorem{theorem}{Теорема}
\newtheorem{corollary}{Следствие}

\newenvironment{cusdefn}[1]
{\renewcommand\thedefn{#1}\defn}
{\enddefn}

\DeclareRobustCommand{\divby}{%
	\mathrel{\text{\vbox{\baselineskip.65ex\lineskiplimit0pt\hbox{.}\hbox{.}\hbox{.}}}}%
}
%Коротки минус
\DeclareMathSymbol{\SMN}{\mathbin}{AMSa}{"39}

\newcommand{\smallerrel}[1]{\mathrel{\mathpalette\smallerrelaux{#1}}}
\newcommand{\smallerrelaux}[2]{\raisebox{.1ex}{\scalebox{.75}{$#1#2$}}}

\newcommand{\smallin}{\smallerrel{\in}}
\newcommand{\smallnotin}{\smallerrel{\notin}}

\newcommand*{\medcap}{\mathbin{\scalebox{1.25}{\ensuremath{\cap}}}}%
\newcommand*{\medcup}{\mathbin{\scalebox{1.25}{\ensuremath{\cup}}}}%

%Подпись символов снизу
\newcommand{\ubar}[1]{\underaccent{\bar}{#1}}

\begin{document}
\lhead{Математический анализ - I}
\chead{Шапошников С.В.}
\rhead{Лекция - 25}

\section*{Формула Тейлора. Ряды Тейлора}

\begin{theorem}
	Если $f$ $n$-раз дифференцируема в точке $a$, то 
	$$f(x) = \sum\limits_{k = 0}^{n}\dfrac{f^{(k)}(a)(x-a)^k}{k!} + \alpha(x){\cdot}(x-a)^n \wedge \lim\limits_{x \to a} \alpha(x) = 0$$
\end{theorem}
\begin{proof}
	Рассмотрим функцию $g(x) = f(x) - \sum\limits_{k = 0}^{n}\tfrac{f^{(k)}(a)(x-a)^k}{k!}, \, g(a) = g^\prime(a) = \dotsc = g^{(n)}(a) = 0$. Надо доказать, что $\lim\limits_{x \to a}\tfrac{g(x)}{(x-a)^n} = 0$. В пределе неопределенность $\tfrac{0}{0}\Rightarrow$ поскольку функция $f$ $n$-раз дифференцируема, то $(n-1)$ производная будет определена в некоторой окрестности точки $a$ и мы используем теорему Лопиталя до порядка $(n-1)$, а затем обычное определение производной в точке $a$:
	$$\lim\limits_{x \to a}\dfrac{g(x)}{(x-a)^n} = \lim\limits_{x \to a}\dfrac{g^\prime(x)}{n(x-a)^{n-1}} = \dotsc = \lim\limits_{x \to a}\dfrac{g^{(n-1)}(x)}{n!(x-a)} = \lim\limits_{x \to a}\dfrac{g^{(n-1)}(x) -g^{(n-1)}(a) }{n!(x-a)}= \dfrac{1}{n!}g^{(n)}(a) = 0$$
\end{proof}
\begin{rem}
	Функция $n$ раз дифференцируема в точке $a \Leftrightarrow$ в окрестности точки $a$ существуют все производные до $(n-1)$-го порядка и $(n-1)$-ая производная в точке $a$ - дифференцируема.
\end{rem}

\subsection*{$O$-символика и $o$-символика}
Пусть функции $f$ и $g$ определены в проколотой окрестности точки $a$ $(\MU^\prime(a))$. 
\begin{defn}
	\textbf{$\bar{o}$-малое}: Если $f(x) = h(x){\cdot}g(x), \, \forall x \in \MU^\prime(a) \wedge \lim\limits_{x\to a}h(x) = 0 \Rightarrow f(x) = \bar{o}(g(x))$, при $x \to a$.
\end{defn}

\begin{defn}
	\textbf{$\ubar{O}$-большое}: Если $f(x) = h(x){\cdot}g(x), \, \forall x \in \MU^\prime(a) \wedge \exists \, C >0 \colon |h(x)| \leq C, \, \forall x \in \MU^\prime(a) \Rightarrow$ \\
	$\Rightarrow f(x) = \ubar{O}(g(x))$, при $x \to a$.
\end{defn}

\begin{rem}
	Если хотим сказать, что функция $f$ получилась из функции $g$, \uline{умножением на ограниченную функцию} $\Rightarrow$ используем $\ubar{O}$-символику.
	
	Если хотим сказать, что функция $f$ получилась из функции $g$, \uline{умножением на функцию стремящуюся к нулю} $\Rightarrow$ используем $\bar{o}$-символику.
\end{rem}

\textbf{Пример}: $f(x) = x^3, \, g(x) = x^2, \, a = 0 \Rightarrow f = \bar{o}(g)$ при $x \to 0$, так как $f(x) = x{\cdot}g(x)$, где $x \to 0$. Надо заметить, что $g \neq\bar{o}(f)$.

\begin{defn}
	Утверждение последней теоремы записывается так:
	$$f(x) = \sum\limits_{k = 0}^{n}\dfrac{f^{(k)}(a)(x-a)^k}{k!} + \underset{x \to a}{\bar{o}((x-a)^n)}$$
	этот вид называется \uwave{формулой Тейлора с остаточным членом в форме Пеано}.
\end{defn}

\textbf{Пример}: $f(x) = e^x, \, a= 0$; Производные экспоненты\\
$(e^x)^{(n)} = e^x \Rightarrow$ 
$$e^x = \sum\limits_{k = 0}^{n}\dfrac{x^k}{k!} + \underset{x \to 0}{\bar{o}(x^n)}$$

\textbf{Пример}: $f(x) = \sin{x}, \, a= 0$; Производные синуса: \\
$(\sin{x})^{(0)} = \sin{x}, \, (\sin{x})^\prime = \cos{x}, \, (\sin{x})^{\prime\prime} = -\sin{x}, \, (\sin{x})^{(3)} = -\cos{x}, \, (\sin{x})^{(4)} = \sin{x} \Rightarrow $ \\
$$\sin{x} = \sum\limits_{k = 0}^{n}\dfrac{(-1)^{k}x^{2k+1}}{(2k+1)!} + \underset{x \to 0}{\bar{o}(x^{2n+2})}$$

\textbf{Пример}: $f(x) = \cos{x}, \, a= 0$; Производные косинуса: \\
$(\cos{x})^{(0)} = \cos{x}, \, (\cos{x})^\prime = -\sin{x}, \, (\cos{x})^{\prime\prime} = -\cos{x}, \, (\cos{x})^{(3)} = \sin{x}, \, (\cos{x})^{(4)} = \cos{x}\Rightarrow$\\  
$$\cos{x} = \sum\limits_{k = 0}^{n}\dfrac{(-1)^{k}x^{2k}}{(2k)!} + \underset{x \to 0}{\bar{o}(x^{2n+1})}$$

\textbf{Пример}: $f(x) = \ln{(1 + x)}, \, a= 0$; Производные логарифма: \\
$(\ln{(1+x)})^{(0)} = \ln{(1+x)}, \, (\ln{(1+x)})^\prime = \dfrac{1}{1+x}, \, (\ln{(1+x)})^{(n)} = \dfrac{(-1)^{n+1}(n-1)!}{(1+x)^n} \Rightarrow$ 
$$\ln{(1+x)} = \sum\limits_{k = 1}^{n}\dfrac{(-1)^{k+1}(k-1)! x^k}{k!} + \underset{x \to 0}{\bar{o}(x^n)} = \sum\limits_{k = 1}^{n}\dfrac{(-1)^{k+1} x^k}{k} + \underset{x \to 0}{\bar{o}(x^n)}$$

\textbf{Пример}: $f(x) = (1 + x)^\alpha, \, \alpha \in \MR, \, a = 0$; Производные логарифма: \\
$((1 + x)^\alpha)^{(0)} = (1 + x)^\alpha, \, ((1 + x)^\alpha)^\prime = \alpha(1+x)^{\alpha-1}, \, ((1 + x)^\alpha)^{(n)} = \alpha(\alpha-1){\cdot}\dotsc{\cdot}(\alpha - n + 1)(1+x)^{\alpha-n} \Rightarrow$ 
$$(1 + x)^\alpha = 1 + \alpha x + \tfrac{\alpha(\alpha-1)}{2}x^2 + \dotsc + \tfrac{\alpha(\alpha-1)\dotsc(\alpha -n + 1)}{n!}x^n + \underset{x \to 0}{\bar{o}(x^n)}, \, \alpha \in \MR$$

Мы пока мало, что знаем про остаточный член в формуле Тейлора. Хотелось бы выяснить его более понятный вид.

\begin{theorem}\textbf{(формула Тейлора с остаточным членом в общей форме)}
	Пусть $f$ $n$-раз дифференцируема в каждой точке отрезка $[a,x]$. Функция $f^{(n)}$ - непрерывна на $[a,x]$ и дифференцируема на интервале $(a,x)$. Пусть функция $g$ - непрерывна на отрезке $[a,x]$ и дифференцируема на интервале $(a,x)$, причем $g^\prime \neq 0$ на $(a,x)$. Тогда $\exists \, c \in (a,x)$:
	$$f(x) = \sum\limits_{k = 0}^{n}\dfrac{f^{(k)}(a)(x-a)^k}{k!} + r_n(x,a)\text{, где } r_n(x,a) = \dfrac{f^{(n+1)}(c)(g(x) - g(a))(x-c)^n}{n! g^\prime(c)} $$
\end{theorem} 
\begin{rem}
	Порядок $[a,x]$ здесь не важен, то есть возможно как $[a,x]$, так и $[x,a]$. 
\end{rem}
\uline{\textbf{Идея}}: Применить теорему Коши к функциям $g(x)$ и $F(t) = f(x) - \displaystyle \sum\limits_{k = 0}^{n}\dfrac{f^{(k)}(t)(x-t)^k}{k!}$.
\begin{proof}
	Пусть $a,\, x$ - зафиксированы, $t \in [a,x] \vee t \in (a,x)$ - меняется. $F(t) = f(x) - \displaystyle\sum\limits_{k = 0}^{n}\dfrac{f^{(k)}(t)(x-t)^k}{k!}$. Функция $f(x)$ $n$ раз дифференцируема и её производная, вплоть до $n$-го порядка, непрерывна на отрезке $[a,x] \Rightarrow F(t)$ - непрерывна на отрезке $[a,x]$. Внутри, на интервале $(a,x)$, у функции $f$ есть производная до $(n+1)$-го порядка $\Rightarrow$ можно продифференцировать $F(t)$ по $t$. Тогда по теореме Коши: 
	$$\exists \, c \in (a,x)\colon \dfrac{F(x) - F(a)}{g(x) - g(a)} = \dfrac{F^\prime(c)}{g^\prime(c)}$$	
	Заметим, что $F(x) = f(x) - f(x) = 0, \, F(a) = f(x) - \sum\limits_{k = 0}^{n}\dfrac{f^{(k)}(a)(x-a)^k}{k!}$, тогда:
	$$f(x) - \sum\limits_{k = 0}^{n}\dfrac{f^{(k)}(a)(x-a)^k}{k!} = -\dfrac{F^\prime(c)(g(x) - g(a))}{g^\prime(c)}$$
	Найдем $F^\prime(c)$:
	$$F^\prime(t) = -f^\prime(t) - f^{\prime\prime}(t)(x-t)  + f^\prime(t) - \dfrac{f^{\prime\prime\prime}(t)(x-t)^2}{2!} + f^{\prime\prime}(t)(x-t) - \dotsc = -\dfrac{f^{(n+1)}(t)(x-t)^n}{n!}$$
	Подставляя полученный результат, получим требуемое:
	$$f(x) - \sum\limits_{k = 0}^{n}\dfrac{f^{(k)}(a)(x-a)^k}{k!} = -\dfrac{F^\prime(c)(g(x) - g(a))}{g^\prime(c)} = \dfrac{f^{(n+1)}(c)(x-c)^n(g(x) - g(a))}{n!g^\prime(c)}$$
\end{proof}

Получив формулу Тейлора с остаточным членом в общей форме, хочется поподставлять разные функции $g(x)$. Рассмотрим конкретные случаи функции $g(x)$.

\begin{corollary}\textbf{(остаточный член в форме Коши)}
	$$g(t) = x -t \Rightarrow r_n(a,x) = \dfrac{f^{(n+1)}(c)(x-c)^n(x-a)}{n!}$$
\end{corollary}
\begin{proof}
	$g^\prime(t) = -1, \, g(x) = 0, \, g(a) = x - a \Rightarrow$ 
	$$r_n(a,x) = \dfrac{f^{(n+1)}(c)(x-c)^n(0-(x-a))}{-n!} = \dfrac{f^{(n+1)}(c)(x-c)^n(x-a)}{n!}$$
\end{proof}

\begin{corollary}\textbf{(остаточный член в форме Лагранжа)}\\
	$$g(t) = (x - t)^{n+1} \Rightarrow r_n(a,x) = \dfrac{f^{(n+1)}(c)(x-a)^{n+1}}{(n+1)!}$$
\end{corollary}
\begin{proof}
	$g^\prime(t) = (n+1)(x-t)^n(-1), \, g(x) = 0, \, g(a) = (x - a)^{n+1} \Rightarrow$ $$r_n(a,x) = \dfrac{f^{(n+1)}(c)(x-c)^n(0-(x-a)^{n+1})}{-n!(n+1)(x-c)^n} = \dfrac{f^{(n+1)}(c)(x-a)^{n+1}}{(n+1)!}$$
\end{proof}

\textbf{Пример}: $f(x) = e^x, \, a= 0$,  Рассмотрим остаточный член в форме Лагранжа: 
$$e^x = \sum\limits_{k = 0}^{n}\dfrac{x^k}{k!} + \dfrac{e^c x^{n+1}}{(n+1)!}, \, c \in (0,x) \vee c \in (x,0)$$
где $\displaystyle \sum\limits_{k = 0}^{n}\dfrac{x^k}{k!}$ - частичная сумма. Как себя ведет остаточное слагаемое, при $n\to \infty$?

$\forall x, \, \bigg|\dfrac{e^c x^{n+1}}{(n+1)!}\bigg| \leq \dfrac{e^{|x|}|x|^{n+1}}{(n+1)!} \to 0$. Тогда $\forall x$ ряд $\sum\limits_{k = 0}^{n}\dfrac{x^k}{k!}$ сходится к $e^x$ (то есть сумма этого ряда в точности равна $e^x$). 

Более того, на всяком отрезке это стремление будет равномерным: $$x \in [-A,A] \Rightarrow \bigg|\dfrac{e^c x^{n+1}}{(n+1)!}\bigg| \leq \dfrac{e^{A}A^{n+1}}{(n+1)!} \Rightarrow \sup\limits_{x \in [-A,A]}\bigg|e^x - \sum\limits_{k = 0}^{n}\dfrac{x^k}{k!}\bigg| \leq \dfrac{e^{A}A^{n+1}}{(n+1)!} \xrightarrow[n\to\infty]{} 0$$

Таким образом $e^x = \displaystyle \sum\limits_{k = 0}^{\infty}\dfrac{x^k}{k!}$ и ряд сходится равномерно на всяком отрезке $[-A,A]$.

\textbf{Пример}: $f(x) = \ln(1+x), \, a= 0$,  Рассмотрим остаточный член в форме Лагранжа: 
$$\ln(1+x) = \sum\limits_{k = 0}^{n}\dfrac{(-1)^{k+1} x^k}{k} + \dfrac{(-1)^{n+2}n! x^{n+1}}{(n+1)!(1+c)^{n+1}} = \sum\limits_{k = 0}^{n}\dfrac{(-1)^{k+1} x^k}{k} + \dfrac{(-1)^{n}x^{n+1}}{(n+1)(1+c)^{n+1}}, \, c \in (0,x) \vee c \in (x,0)$$ 
 
Если $x > 1$, то ряд $ \displaystyle \sum\limits_{k = 0}^{n}\dfrac{(-1)^{k+1} x^k}{k}$ расходится.

Если $0 < x \leq 1$, тогда $c \in(0,x), \, c > 0, \, (1 + c) > 1  \Rightarrow \bigg|\dfrac{(-1)^{n}x^{n+1}}{(n+1)(1+c)^{n+1}}\bigg| \leq \dfrac{1}{n+1} \to 0$.  Тогда $\forall x$ ряд $\displaystyle \sum\limits_{k = 0}^{n}\dfrac{(-1)^{k+1} x^k}{k}$ сходится к $\ln(1+x)$ (то есть сумма этого ряда в точности равна $\ln(1+x)$). 

Если $-1 < x < 0$, тогда $c \in (x,0),\, c < 0$, но мы не можем сказать как соотносится $x$ и $c - (-1) = 1+c \Rightarrow$ также не можем ничего сказать про остаточный член $\dfrac{(-1)^n}{n+1}\bigg( \dfrac{x}{1+c}\bigg)^{n+1} \Rightarrow$ рассмотрим остаточный член в форме Коши:
$$\dfrac{(-1)^n n!}{(1+c)^{n+1}}\dfrac{(x-c)^n x}{n!} = \dfrac{(-1)^n (x - c)^n x}{(1+c)^{n+1}} = \dfrac{(-1)^n (x - c)^n}{(1+c)^{n}}{\cdot}\dfrac{x}{(1+c)}$$
$c > x \Rightarrow 1+c > 1 + x \Rightarrow \dfrac{x}{1+c}\leq \dfrac{|x|}{|1+x|}$, так как $c< 0 \wedge x <0 \wedge 1 > |x| > |c| \Rightarrow \bigg|\dfrac{x-c}{1+c} \bigg| = \dfrac{|x| - |c|}{1 - |c|} = 1 - \dfrac{1 - |x|}{1-|c|} \leq 1 - (1-|x|) = |x| \Rightarrow
$ оценим остаточный член:
$$
\bigg|\dfrac{(-1)^n (x - c)^n x}{(1+c)^{n+1}}\bigg|  = \bigg|\dfrac{(-1)^n (x - c)^n}{(1+c)^{n}}{\cdot}\dfrac{x}{(1+c)}\bigg|\leq \dfrac{|x|}{|1+x|}|x|^n \xrightarrow[n \to \infty]{} 0
$$

Таким образом $\forall x \in (-1,1], \, \ln{(1+x)} = \displaystyle \sum\limits_{k = 0}^{\infty}\dfrac{(-1)^{k+1} x^k}{k}$

\begin{rem}
	Ряд Лейбница: $\ln{2} =\displaystyle \sum\limits_{k = 0}^{\infty}\dfrac{(-1)^{k+1}}{k}$. Ряд сходится, но не сходится ряд из его модулей.
\end{rem}

\end{document}